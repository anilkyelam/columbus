\section{RELATED WORK}
\label{sec:relatedwork}
\todo{needs polishing; make sure that terms are consistent with the work,
covert-channel vs side-channel}

\textbf{Covert Channels \& Cloud Attacks} Co-residency is possible because of
covert channels, so we begin our related work with an investigation into covert
channels and cloud attacks. Initial papers in co-residency detection utilized
host information and network addresses arising due to imperfect
virtualization~\cite{ristenpartccs2009}. However, these covert channels are now
obsolete, as cloud provides have strengthened virtualization and introduced
Virtual Private Clouds\todo{cite}. Later work used cache-based channels in
various levels of the cache~\cite{xuccsw2011, zhangccs2014, liu2015,
kaylaap2016} and hardware based covert channels like thermal covert channels
\cite{mastiusenix2015}, RNG module \cite{evtyushkinccs2016} and memory bus
\cite{wuusenix2012} have also been  explored in the recent past. Moreover,
studies have found that VM performance can be significantly degraded using
memory DDoS attacks~\cite{zhang2016memory}, while containers are susceptible to
power attacks from adjacent containers~\cite{gao2017}. Our work focuses on using
the memory bus as a covert channel for determining cooperative co-residency.
.\amirian{work this in?}Covert channels using memory bus were first introduced
by Wu etl. al~\cite{whispers}, and subsequently has been used for co-residency
detection on VMs and Containers~\cite{compstudycoresidency,varad191016}  Wu et.
al~\cite{whispers} introduced a new technique to lock the memory bus by using
atomic memory operations on addresses that fall on multiple cache lines. .  \\

\textbf{Co-residency} One of the first pieces of literature in detecting VM
co-residency was introduced by Ristenpart et al., who demonstrated that VM
co-residency detection was possible and that these techniques could be used to
gather information about the victim machine (such as keystrokes and network
usage)~\cite{ristenpartccs2009}. This initial work was further expanded in
subsequent years to examine co-residency using memory bus
locking~\cite{xuusenix2015} and active traffic analysis~\cite{bates2012}, as
well as determining placement vulnerabilities in multi-tenant Public
Platform-as-a-Service systems~\cite{varadarajan2015, zhangpaas2016}. Finally,
Zhang et al. demonstrated a technique to detect VM co-residency detection via
side-channel analyses~\cite{zhang2011}. Our work expands on these previous works
by investigating co-residency for lambdas.\\

\textbf{Lambdas} \todo{pretty rough paragraph...needs a rewrite} While lambdas
are a newer technology than VMs, there still exists a variety of literature. For
example, recent studies examine cost comparisons of running web applications in
the cloud on lambdas versus other architectures~\cite{villamizar2016}. Moreover,
lambdas have been studied in the context of cost-effective batching and data
processing~\cite{kiran2015}. Further research has shown how lambdas perform with
scalability and hardware isolation, indicating some flaws in the lambda
architecture~\cite{wangusenix2018}. From a security perspective, Izhikevich et.
al examined lambda co-residency using RNG and memory bus techniques (similar to
techniques used  looking at VM co-residency)~\cite{izhikevich2018}.. However,
our work differs from this study in that our technique informs the user of which
lambdas are on the same machine, not only that the lambdas experience
co-residency.
