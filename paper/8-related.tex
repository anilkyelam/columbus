\section{RELATED WORK}
\label{sec:relatedwork}

\noindent \textbf{Cloud Attacks} 
Co-residency is possible because of covert channels, so we begin our related
work with an investigation into cloud attacks. Initial papers in co-residency
detection utilized host information and network addresses arising due to
imperfect virtualization~\cite{ristenpartccs2009}.  However, these channels are
now obsolete, as cloud provides have strengthened virtualization and introduced
Virtual Private Clouds~\cite{awsvpc}. Later work used cache-based channels in
various levels of the cache~\cite{xuccsw2011, zhangccs2014, liu2015,
kaylaap2016} and hardware based channels like thermal covert
channels~\cite{mastiusenix2015}, RNG module~\cite{evtyushkinccs2016} and memory
bus~\cite{wuusenix2012} have also been explored in the recent past. Moreover,
studies have found that VM performance can be significantly degraded using
memory DDoS attacks~\cite{zhang2016memory}, while containers are susceptible to
power attacks from adjacent containers~\cite{gao2017}.  

Our work focuses on using the memory bus as a covert channel for determining
cooperative co-residency.  Covert channels using memory bus were first
introduced by Wu et. al~\cite{wuusenix2012}, and subsequently has been used for
co-residency detection on VMs and containers~\cite{zhang2011,varad191016} Wu
et. al~\cite{wuusenix2012} introduced a new technique to lock the memory bus by
using atomic memory operations on addresses that fall on multiple cache lines, a
technique we rely on in our own work. \\

\noindent \textbf{Co-residency} 
One of the first pieces of literature in detecting VM co-residency was
introduced by Ristenpart et al., who demonstrated that VM co-residency detection
was possible and that these techniques could be used to gather information about
the victim machine (such as keystrokes and network
usage)~\cite{ristenpartccs2009}. This initial work was further expanded in
subsequent years to examine co-residency using memory bus
locking~\cite{xuusenix2015} and active traffic analysis~\cite{bates2012}, as
well as determining placement vulnerabilities in multi-tenant Public
Platform-as-a-Service systems~\cite{varadarajan2015, zhangpaas2016}. Finally,
Zhang et al. demonstrated a technique to detect VM co-residency detection via
side-channel analyses~\cite{zhang2011}. Our work expands on these previous works
by investigating co-residency for lambdas.\\

\noindent \textbf{Lambdas} 
While lambdas are a much newer technology than VMs, there still exists
literature on the subject. Recent studies examined cost comparisons of running
web applications in the cloud on lambdas versus other
architectures~\cite{villamizar2016}, and also examined the lambdas have been
studied in the context of cost-effectiveness of batching and data processing
with lambdas~\cite{kiran2015}.  Further research has shown how lambdas perform
with scalability and hardware isolation, indicating some flaws in the lambda
architecture~\cite{wangusenix2018}. From a security perspective, Izhikevich et.
al examined lambda co-residency using RNG and memory bus techniques (similar to
techniques utilized in VM co-residency)~\cite{izhikevich2018}. However, our work
differs from this study in that our technique informs the user of which lambdas
are on the same machine, not only that the lambdas experience co-residency.
