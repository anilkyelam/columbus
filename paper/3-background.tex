\section{Background}
\label{sec:background}
We begin with a brief background on related topics.

\subsection{Lambdas/Serverless Functions} 
\label{sec:background:lambdas}

We focus on serverless functions in this paper, as they are one of the
fastest-growing cloud services and are less well-studied from a security
standpoint. Offered as lambdas on AWS~\cite{awslambda}, and as cloud functions
on GCP~\cite{gcpfunctions} and Azure~\cite{azurefunctions}, these functions are
of interest because they do not require the developer to provision, maintain, or
administer servers. In addition to this low maintenance, lambdas are much more
cost-efficient than virtual machines (VMs) as they allow more efficient packing
of functions on servers. Moreover, lambdas execute as much smaller units  and
are more ephemeral than virtual machines.  For example, on AWS,
the memory of lambdas is capped at 3 GB, with a maximum execution limit of 15
minutes.  As with other cloud services, the user has no control over the
physical location of the server(s) on which their lambdas are spawned.

While lambdas are limited in the computations they can execute (typically
written in high-level languages like Python, C\#, etc), they are conversely
incredibly lightweight and can be initiated and deleted in a very short amount
of time. Cloud providers run lambdas in dedicated containers with limited
resources (e.g., Firecracker ~\cite{firecracker}), which are usually cached and
re-used for future lambdas to mitigate cold-start
latencies~\cite{awscontainerreuse}. The ephemeral nature of serverless functions
and their limited flexibility increases the difficulty in detecting
co-residency, as we will discuss later. While previous studies that profiled
lambdas~\cite{wangusenix2018} focused on the performance aspects like cold start
latencies, function instance lifetime, and CPU usage across various clouds,
the security aspects remain relatively understudied.


\subsection{Covert Channels in the Cloud}
\label{sec:background:covertchannels}
In our attempt to shed light on the security aspects of lambdas, we focus
particularly on the feasibility of establishing a reliable covert channel in the
cloud using lambdas.  Covert channels enable a means of transmitting information
between entities that bypasses traditional monitoring or auditing. Typically,
this is achieved by communicating data across unintended channels such as
signaling bits by causing contention on shared hardware media on the
server~\cite{L2cacheCovertChannels,
ProcessorCovertChannels,ThermalCovertChannel,SshOverCovertChannel,wuusenix2012}.
Past work has demonstrated covert channels in virtualized environments like the
clouds using various hardware such as
caches~\cite{ristenpartccs2009,L2cacheCovertChannels}, memory
bus~\cite{wuusenix2012}, and even processor
temperature~\cite{ThermalCovertChannel}. \\

% \noindent \textbf{Memory bus covert channel} Alex's feedback
Of particular interest to this work is the covert channel based on memory bus
hardware introduced by Wu et al.~\cite{wuusenix2012}.  In x86 systems, atomic
memory instructions designed to facilitate multi-processor synchronization are
supported by cache coherence protocols as long as the operands remain within a
cache line (generally the case as language compilers make sure that operands are
aligned). However, if the operand is spread across two cache lines (referred to
as "exotic" memory operations), x86 hardware achieves atomicity by locking the
memory bus to prevent any other memory access operations until the current
operation finishes. This results in significantly higher latencies for such
locking operations compared to traditional memory accesses. As a result, a few
consecutive locking operations could cause contention on the memory bus that
could be exploited for covert communication.  Wu et al. achieved a data rate of
700 bits per second (bps) on the memory bus channel in an ideal laboratory
setup.

Achieving such ideal performance, however, is generally not possible in cloud 
environments. Cloud platforms employ virtualization to enable statistical 
multiplexing and as such, communication on the covert channel may be affected by:
1) scheduler interruptions as the sender or the receiver may only get intermittent 
access to the channel and 2) interference from other non-participating
workloads. This may result in errors in the transmitted data and require 
additional mechanisms like error correction~\cite{wuusenix2012} to ensure
reliable communication.


\subsection{Co-residence Detection}
\label{sec:background:coresidence}

%On previous work on colocation, past techniques and why those techniques would 
%not work anymore
In the cloud context, enabling communication over traditional covert channels 
comes with an additional challenge of placing sender and 
receiver on the same machine. However,
such co-residency information is hidden, even if the entities belong to the same
tenant. Past research has used various strategies to achieve co-residency in order 
to demonstrate various covert channel attacks in the cloud. Typically, the
attacker launches a large number of cloud instances (VMs, Lambdas, etc.),
following a certain launch pattern, and employs a co-residence detection
mechanism for detecting if any pair of those instances are running on the same
machine. Traditionally, such detection has been based on software
runtime information that two instances running on the same server might share,
like public/internal IP addresses~\cite{ristenpartccs2009}, files in
\textit{procfs} or other environment variables~\cite{wangusenix2018,wuusenix2012}, 
and other such logical side-channels~\cite{varad191016,vmplacement}.

As virtualization platforms moved towards stronger isolation between instances
(e.g.  AWS' Firecracker VM~\cite{firecracker}), these logical covert-channels
have become less effective or infeasible. Furthermore, some of these channels
were only effective on container-based platforms that shared the underlying OS
image and were thus less suitable for hypervisor-based platforms.  This prompted
a move towards using hardware-based covert channels, such as the ones discussed
in section \ref{sec:background:covertchannels}, which can bypass software
isolation and are thus usually harder to fix. For example, Varadarajan et
al.~\cite{varadarajan2015} use the memory bus covert channel to detect
co-residency for EC2 instances.  However, as we will show, their approach does
not extend well to lambdas as it is neither fast nor scalable.

