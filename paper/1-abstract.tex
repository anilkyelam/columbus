
\begin{abstract}


\amirian{Cloud computing has seen explosive growth in the past decade. While
efficient sharing of infrastructure among tenants has contributed to this
growth, the same principles also open avenues for covert-channels, or the
capability to share information between processes that should be hypothetically
isolated.  Providers like AWS and Azure have traditionally relied on obfuscation
to prevent this leakage of information, but recent works have repeatedly found
detection techniques that break this encapsulation, prompting the providers to
harden isolation on their platforms. In this work, we find yet another such
covert-channel for lambdas based on the memory bus, which is more pervasive,
reliable, and harder to fix.  We show that this technique can be used to
reliably perform co-operative co-residence detection for thousands of AWS
lambdas within a few seconds, which could aid attackers in performing DDoS
attacks or learn cloud's internal mechanisms. In this paper, we present the
technique in detail, evaluate it, and use it to perform a measurement study on
lambda activity across AWS regions.  Through this work, we hope to motivate the
need to address this covert channel in the cloud.}


Cloud computing has seen explosive growth in the past
decade. This is made possible by efficient sharing of
infrastructure among tenants, which unfortunately also 
opens avenues for security attacks (e.g., side-channels). Providers, like AWS
and Azure, have traditionally relied on hiding the 
co-residency information to prevent targeted attacks 
in their clouds. But recent works have repeatedly 
found co-residence detection techniques that break
this encapsulation, prompting the providers to address 
them and harden isolation on their platforms. In 
this work, we find yet another such technique based on 
a memory bus covert channel, which 
is more pervasive, reliable and harder to fix. We 
show that we can use this technique to reliably perform co-operative co-residence detection for thousands 
of AWS lambdas within a few seconds, which could aid attackers in performing DDoS attacks or learn 
cloud's internal mechanisms. In this paper, we present the technique 
in detail, evaluate it and use it to perform a small
study on lambda activity across AWS regions. 
Through this work, we hope to motivate the need to 
address this covert channel in the cloud.

% While the providers generally hide the placement information, 
% numerous works in the past have proposed hardware and software-based 
% information leakage channels that allow for co-residency detection 
% for virtual machines and containers. In this paper, we morph previously-known 
% hardware-based channels to create a methodology that allows for 
% cooperative co-residency detection for serverless functions. We use these 
% channels to successfully achieve co-location in both AWS and GCP clouds, 
% and evaluate each of the channels for their efficiency in different 
% scenarios using different regions in these cloud services.

% we intend to highlight how easy it is and further motivate cloud providers 
% to address the covert channel
\end{abstract}
