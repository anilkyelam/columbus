
\begin{abstract} 

``Serverless'' cloud services, such as AWS lambdas, are one of the fastest
growing segments of the cloud services market. These services are popular in
part due to their light-weight nature and flexibility in scheduling and cost,
however the security issues associated with serverless computing are not well
understood. In this work, we explore the feasibility of constructing a practical
covert channel from lambdas.  \amirian{Suggested change: We explore the
challenges of a co-residence detector for lambdas and proceed to develop a
reliable and scalable co-residence detector using the memory bus hardware.}
\anil{prefer not, because that adds a jump to co-residece detector from covert channels, 
reader may not quickly connect the two} We
establish that a fast co-residence detection for lambdas is key to enabling such
a covert channel, and proceed to develop a reliable and scalable co-residence
detector based on the memory bus hardware. Our technique enables dynamic
discovery for co-resident lambdas and is incredibly fast, executing in a matter
of seconds.  We evaluate our approach for correctness and scalability, and use
it to establish covert channels and perform data transfer on AWS lambdas.  We
show that we can establish tens of individual covert channels for every 1000
lambdas, and each of those channels can send data at a rate of ~200 bits per
second, thus demonstrating the feasibility of covert communication via lambdas.
%\ugh{Through this work, we show that efforts to secure 
%co-residency detection on cloud platforms are not yet complete.}

%Cloud computing has seen explosive growth in the past decade. While
%efficient sharing of the underlying server infrastructure among tenants 
%has contributed to this growth, the same principles also open avenues for 
%cross-tenant attacks. Providers like AWS and Azure have 
%traditionally relied on obfuscation of coresidency information 
%to prevent such attacks. However, recent works have repeatedly found
%covert channels that allow for breaking this encapsulation and detecting
%co-residency, prompting the providers to harden isolation on their platforms. 
%In this work, we find yet another such technique for detecting coresidency of 
%cloud instances. Our technique uses a hardware covert channel based on the 
%memory bus, which is more pervasive and harder to fix. 
%We show that this channel can be used to perform reliable co-residence detection 
%% reliable and fast go against each other. just doing reliable is easy, as you can 
%% get 100% reliability by letting it run forever. The challenge is to have 
%% something that is fast and yet reliable. I don't know how to convey that nicely 
%one that is fast enough for detecting coresidency for the ephemeral serverless functions,
%or lambdas. We use our technique to perform co-operative co-residence detection 
%for thousands of AWS lambdas within a few seconds, which could aid attackers in 
%performing DDoS attacks or learn cloud's internal mechanisms. 
%In this paper, we present the technique in detail, evaluate it, 
%and use it to perform a measurement study on
%lambda activity across AWS regions.  Through this work, we hope to motivate the
%need to address this covert channel.
\end{abstract}
