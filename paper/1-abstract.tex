
\begin{abstract}
Cloud computing has seen explosive growth in the past decade. While
efficient sharing of infrastructure among tenants has contributed to this
growth, the same principles also open avenues for covert-channels, or the
capability to share information between processes that should be hypothetically
isolated.  Providers like AWS and Azure have traditionally relied on obfuscation
to prevent this leakage of information, but recent works have repeatedly found
detection techniques that break this encapsulation, prompting the providers to
harden isolation on their platforms. In this work, we find yet another such
covert-channel for lambdas based on the memory bus, which is more pervasive,
reliable, and harder to fix.  We show that this channel can be used to
reliably perform co-operative co-residence detection for thousands of AWS
lambdas within a few seconds, which could aid attackers in performing DDoS
attacks or learn cloud's internal mechanisms. In this paper, we present the
technique in detail, evaluate it, and use it to perform a measurement study on
lambda activity across AWS regions.  Through this work, we hope to motivate the
need to address this covert channel.
\end{abstract}
