
\begin{abstract}
Cloud computing has seen explosive growth in the past decade. While
efficient sharing of the underlying server infrastructure among tenants 
has contributed to this growth, the same principles also open avenues for 
cross-tenant attacks. Providers like AWS and Azure have 
traditionally relied on obfuscation of coresidency information 
to prevent such attacks. However, recent works have repeatedly found
covert channels that allow for breaking this encapsulation and detecting
co-residency, prompting the providers to harden isolation on their platforms. 
In this work, we find yet another such technique for detecting coresidency of 
cloud instances. Our technique uses a hardware covert channel based on the 
memory bus, which is more pervasive and harder to fix. 
We show that this channel can be used to perform reliable co-residence detection 
% reliable and fast go against each other. just doing reliable is easy, as you can 
% get 100% reliability by letting it run forever. The challenge is to have 
% something that is fast and yet reliable. I don't know how to convey that nicely 
one that is fast enough for detecting coresidency for the ephemeral serverless functions,
or lambdas. We use our technique to perform co-operative co-residence detection 
for thousands of AWS lambdas within a few seconds, which could aid attackers in 
performing DDoS attacks or learn cloud's internal mechanisms. 
In this paper, we present the technique in detail, evaluate it, 
and use it to perform a measurement study on
lambda activity across AWS regions.  Through this work, we hope to motivate the
need to address this covert channel.
\end{abstract}
