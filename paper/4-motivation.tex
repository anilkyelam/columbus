
\section{Motivation}
\label{sec:motivation}

In this section, we discuss our covert channel attack scenario, the challenges
lambdas would pose in enabling such an attack and, the need for a co-residence
detector for lambdas.

\textbf{Threat Model}
Covert channel attacks generally require an ``insider'' who sends data over a
covert medium for exfiltration. We assume that the attacker uses social
engineering techniques or some other means (beyond the scope of this work) to
introduce  such insiders in the victim system. In the case of lambdas, this
insider code could be in the lambda itself or in a system that controls lambda
deployments for an organization and already possesses the sensitive data that
needs to be exfiltrated.  We further assume that the attacker has the knowledge
of the cloud region where the victim is operating and can deploy lambdas under
its own account.

In a typical attack, the attacker launches a set of lambdas (receivers) in the
cloud region where the victim lambdas (senders) are expected to operate. The
attacker and (compromised) victim lambdas can then work together\footnote{We
do not explicitly differentiate attack and victim lambdas hereafter as they are
all assumed to be in the control of the attacker} to exchange the data over a covert
channel, like the memory bus hardware discussed earlier.  However, as
mentioned earlier, there are few unique challenges before we can use a
traditional covert channel in the cloud. We need to: 1) co-locate the sender and
receiver on the same server (via co-residence detection) and 2)
handle interruptions on such a channel, such as noisy neighbors
neighbors or scheduling.

While these challenges have been handled for other cloud platforms like 
VMs~\cite{varad191016,wuusenix2012}, lambdas are inherently different in that they have very short
lifetimes. A covert channel between two co-resident lambdas will not last very
long. However, while lambdas are not persistent, it is trivial and cheap to
launch lambdas in large numbers at once and establish multiple co-residence
points to allow for more covert communication.  Additionally, lambdas are also
more densely packed than VMs, exacerbating the noisy neighbor problem.

The ephemeral, numerous, and dense nature of lambdas require a fast, scalable,
and reliable co-residence detector. This detector should also allow the attacker
identify all the servers with two or more co-resident lambdas and establish a
covert channel on each such server. Moreover, the detector should precisely
identify how many and which lambdas are co-resident, allowing the attacker to
pick any two lambdas on a given machine, and set up a covert channel between
these two lambdas without interference from the rest. 
%This means that they can  quickly identify which lambdas are singletons and save
%time and money by killing these functions early.


