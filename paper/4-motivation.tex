
\section{Motivation}
\label{sec:motivation}

We discuss the covert channel attack scenario that we are targetting, the challenges 
lambdas would pose in enabling such an attack and motivate the need for a 
co-residence detector for lambdas which is going to be our main focus in the rest of 
the paper. 

\textbf{Threat Model}
Covert channel attacks require an "insider" to send the data over a covert 
medium for exfiltration. We assume that the attacker uses social engineering 
techniques or some other means (beyond the scope of this work) to introduce  
such insiders in the victim system. In case of lambdas, this insider code could be 
in the lambda itself or in a system that controls lambda deployments for an 
organization; and already possesses the sensitive data that needs to be exfiltrated. 
We further assume that the attacker has the knowledge of the cloud region 
where the victim is operating and can deploy lambdas under its own account.

In a typical attack, the attacker launches a set of  lambdas (receivers) in the 
cloud region where the victim lambdas 
(senders) are expected to operate. The attacker and 
(comprimised) victim lambda(s) can then work together\footnote{We do not explicitly
differentiate attack and victim lambdas hereafter as they are all assumed 
to be in attacker's control} to exchange the data over 
a covert channel like the memory bus hardware discussed in earlier section.
However, as mentioned earlier, there are few unique challenges before we can use
a traditional covert channel in the cloud. We need: 1) to colocate the sender 
and receiver on the same server which requires co-residence detection and 2) to handle
the interruptions on such channel introduced by noisy neighbors and inconsistent 
access to the channel due to scheduling.

While these challenges have been handled for other cloud platforms like 
VMs~\cite{varad191016,wuusenix2012}, 
lambdas are inherently different in that 
they have very short lifetimes. A covert
channel between two co-resident lambdas will not last very long. However, while
lambdas are not persistent, it is trivial and cheap to launch lambdas in large numbers at
once and establish multiple rendezvous points to allow for more covert communication.
Additionally, lambdas are also densely packed than VMs
exacerbating the noisy neighbor problem.

The ephemeral, numerous and dense nature of lambdas require a fast,
scalable, and reliable co-residence detector. Such a  
detector will allow the attacker to identify all the servers with two or more 
co-resident lambdas and establish a rendezvous on each such server. 
Moreover, the detector should precisely identify which lambdas are co-resident 
allowing the attacker to pick any two lambdas on a given machine, 
and use the covert channel without interference from neighbors. 
%This means that they can  quickly identify which lambdas are singletons and save
%time and money by killing these functions early.


