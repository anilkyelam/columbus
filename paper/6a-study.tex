\section{Placement Study}
\label{sec:study}

Our goal in developing this technique is to demonstrate how 
easy it is for attackers to exploit the pervasive memory bus covert 
channel and obtain co-residency information, thereby motivating 
the need to address it. This information can be used by attackers 
in aiding a lot of attack scenarios \todo{for example?} or simply learn 
the internal mechanisms of a cloud. In the section, we explore some 
ways in which the tool can be used to gain some insights into lambda 
activity in some AWS regions. Unless specified otherwise, all the 
experiments are performed with 1.5 GB lambdas and ran successfully with
zero error rate. We find that \todo{summary of takeaways from the study}.

% we have not even kicked off any alarms in the clouds - no notifications.

\subsection{Co-residence across AWS regions}
We ran co-residence detection in different AWS regions with 
1000 1.5GB Lambdas. Figure \ref{fig:awsregions} shows multiple 
plots showing the colocated groups, one per region, with each 
bar in the plot showing the 
fraction of lambdas that saw a certain number of neighbors (i.e.,
that belong to a colocated group of certain size). Plots that 
are right-heavy (towards bottom-right) indicate higher colocation density
compared to the left-heavy (towards top-left) ones (which is also illustrated in 
figure \ref{fig:density}). We can see that most 
regions have almost all lambdas see at least one neighbor (smaller 
or non-existent bar at 0). Assuming that the cloud placement scheduler
tries to efficiently bin pack lambdas, we hypothsize that the 
colocation is dependent on the total number of servers and the lambda
activity in the region, both of which can be assumed to be lower 
in newer AWS regions. We note that most right-heavy plots correspond 
to the relatively newer AWS regions. The maximum size of a 
co-located group we ever saw was 25 (1.5G) lambdas on a single machine.
% max is 25 lambdas
% indicative of lambda activity and size of the region
% newer regions see less activity or less number of servers
\todo{any other insights I missed?}


\subsection{Weekly \& Daily Patterns}
\todo{Run experiments in different times of the day or using 
different deployment strategies that may affect co-location.}


\subsection{Different accounts}
\todo{Run experiments with lambdas from different user accounts 
and see how the colocation is affected between single vs different 
accounts.}