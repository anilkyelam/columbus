\section{Measurement Study}
\label{sec:study}

%Our goal in developing this technique is to demonstrate how easy it is for
%attackers to exploit the pervasive memory bus covert channel and obtain
%co-residency information, thereby motivating the need to address it. This
%information can be used by attackers in aiding a lot of attack scenarios
%\todo{for example?} or simply learn the internal mechanisms of a cloud. 
%In the section, we explore some ways in
%which the tool can be used to gain some insights into lambda activity in some
%AWS regions.  
Next, we present a variety of measurements on AWS that present this approach as
practical, across various regions. Unless specified otherwise, all the
experiments are performed with 1.5 GB lambdas and executed successfully with
\textbf{a zero percent error rate}. We find that \todo{summary of takeaways from the
study}.

% we have not even kicked off any alarms in the clouds - no notifications.

\subsection{Co-residence across AWS regions}
We deployed our co-residence detection in different AWS regions with 1000 1.5GB
Lambdas. Figure~\ref{fig:awsregions} is compromised of multiple plots indicated
the co-resided groups per region, with each bar indicating the fraction of
lambdas that detected a certain number of neighbors (i.e., that belong to a
co-resided group of a certain size). Plots that skew to the right indicate a
higher co-resided density when compared to the plots skewed to the left (also
illustrated in Figure~\ref{fig:density}). We note that, in most regions, almost
all lambdas recognize at least one neighbor.
%almost all lambdas see at least one neighbor (smaller or non-existent bar at 0).
Assuming that the cloud placement scheduler tries to efficiently 
%bin  -- removed this word because we havent defined it and might confused the
%reader
pack lambdas, we hypothsize that the colocation is dependent on the total number
of servers and the lambda activity in the region, both of which can be assumed
to be lower in newer AWS regions. We note that plots skkewed most to the right
correspond to the relatively newer AWS regions.  We note that the largest
co-resided group on a single machine was compromised of 25 lambdas.
% max is 25 lambdas
% indicative of lambda activity and size of the region
% newer regions see less activity or less number of servers
\todo{any other insights I missed?}


\subsection{Weekly \& Daily Patterns}
\todo{Run experiments in different times of the day or using 
different deployment strategies that may affect co-location.}


\subsection{Different accounts}
\todo{Run experiments with lambdas from different user accounts 
and see how the colocation is affected between single vs different 
accounts.}
