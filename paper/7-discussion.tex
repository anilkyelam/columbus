\section{DISCUSSION}
\label{sec:discussion}
\textbf{Alternate use cases}
Our main motivation behind proposing a co-residence detector for lambdas is
to demonstrate the feasibility of a covert channel. However, there are other
scenarios where such tool can be (ab)used, of which we provide a couple of examples. 
\begin{itemize}
%     \item While our co-residence detector does not directly help attackers
%     locate their victims in the cloud, it can aid them in performing devastating 
%     DDOS attacks once by concentrating a number of attack instances on the victim 
%     machine. Also, the attacker could try to gain a wider surface area for 
%     targeted attacks in a cost-effective way by turning on/off her co-resident 
%     instances as necessary. 
    \item Previous studies on performance aspects of lambdas (like performance 
    isolation~\cite{wangusenix2018}) generally need a way to find co-resident
    lambdas. As software-level logical channels begin to disappear, our tool 
    might provide a reliable alternative.
    \item Burst parallel frameworks~\cite{234886} that orchestrate lambdas can
    use our co-residence detector as a locality indicator to take advantage of
    server locality.
\end{itemize}

\textbf{Mitigation}
In the previous section, we showed that our co-residence detector makes the covert
channels practical with lambdas, so it is important that clouds address this
issue. One way to disable our co-residence detector is to fix the underlying
memory bus channel that it employs. However, this only works for newer
generation of servers and is not practical for existing infrastructure. An
easier solution, one that is only practical with lambdas, is to disable the
lambda support for low-level languages (or unsafe versions of high-level
languages) by the cloud providers. This will prevent pointer arithmetic operations that are
required to activate this channel. However, in all three cloud platforms we examined, 
these unsafe or low level languages were introduced as options at a later point, 
indicating that there is a business use case. In that case, cloud providers may look at
more expensive solutions like BusMonitor~\cite{MemoryBusMitigation} that isolates
memory bus usage for different tenants by trapping the atomic operations to the 
hypervisor. We leave such exploration to future work.
