
\section{ETHICAL CONSIDERATIONS}
As with any large scale measurement project, we discuss the ethical
considerations. First, there are security and privacy concerns of using this
technique to uncover other consumer's lambdas. However, since we focus on
co-operative co-residence detection, we only determine co-residence for the
lambdas we launched, and do not gain insight into other consumer's lambdas.
Second, there is the concern that our experiments may cause performance issues with
other lambdas, as we may block their access to the memory bus. We believe this
concern is small, for a number of reasons. Memory accesses are infrequent due to
the multiple levels of caches; we would only be affecting a small number of
operations. Memory accesses and locking operations are FIFO, which prevents
starvation of any one of the lambdas sharing a machine. Moreover, lambdas are
generally not recommended for latency-sensitive workloads, due to their
cold-start latencies. Thus, the small amount of lambdas that we might affect
should not, in practice, be affected in their longterm computational goals. 


\section{CONCLUSION}
\label{sec:conclusion}
In this paper, we have demonstrated a technique to build covert channels 
entirely using serverless cloud functions such as 
AWS lambdas. To achieve this goal, we developed a fast and reliable co-residence 
detector for lambdas, and evaluated it for correctness and scalability.
Finally, we have empirically demonstrated the practicality of such covert 
communication by studying the covert channel capacity and 
co-residence density of lambdas on various AWS regions.


