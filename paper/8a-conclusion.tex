
\section{ETHICAL CONSIDERATIONS}
As with any large scale measurement project, there are ethical considerations to
take into account. First, there are security and privacy concerns of using this
technique to uncover other consumer's lambdas. However, since we focus on
co-operative co-residence detection, we only determine co-residence for the
lambdas we launched, and do not gain insight into other consumer's lambdas.
Second, there is concern that our experiments may cause performance issues
with other lambdas, as we may block their access to the memory bus. We believe
this concern is small, for a number of reasons. Memory accesses are infrequent
due to the multiple levels of caches; we would only be affecting a small number
of operations. Memory accesses and locking operations are FIFO, which
prevents starvation of any one of the lambdas sharing a machine. Moreover,
lambdas are generally not recommended for latency-sensitive workloads, due to
their cold-start latencies. Thus, the small amount of lambdas that we might
affect should not, in practice, be affected in their longterm computational
goals. 


\section{CONCLUSION \& FUTURE WORK}
\label{sec:conclusion}
\todo{Copied abstract} 
Cloud computing has seen explosive growth in the past decade. This is made
possible by efficient sharing of infrastructure among tenants, which
unfortunately also raises security challenges like preventing side-channel
attacks. Providers, like AWS and Azure, have traditionally relied on hiding the
co-residency information to prevent targeted attacks in their clouds. But recent
works have repeatedly found co-residence detection techniques that break this
encapsulation, prompting the providers to address them and harden isolation on
their platforms. In this work, we find yet another such technique based on a
memory bus covert channel that is more pervasive, reliable and harder to fix. We
show that we can use this technique to reliably perform co-operative
co-residence detection for thousands of AWS lambdas within a few seconds, which
opens a way for attackers to perform DDoS attacks or learn cloud's internal
mechanisms. We present this technique in detail, evaluate it and use it to
perform a small study on lambda activity across a few AWS regions.  Through this
work, we hope to motivate the need to address this covert channel in the cloud.

