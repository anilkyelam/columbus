
\section{ETHICAL CONSIDERATIONS}
\todo{Get help from Ariana/Stefan}. We are not violating security or privacy of other 
lambdas in any way whatsoever. There may be some performance interference caused 
by blocking other lambdas access to memory bus but this is minimal given 
that memory accesses are infrequent due to multiple levels of caches; also memory accesses/locking operations are FIFO too which prevents starvation. Also, lambdas are not meant for latency-sensitive workloads due to their cold-start latencies.

\section{CONCLUSION \& FUTURE WORK}
\label{sec:conclusion}
\todo{Copied abstract} 
Cloud computing has seen explosive growth in the past
decade. This is made possible by efficient sharing of
infrastructure among tenants, which unfortunately also raises security challenges like preventing side-channel attacks. Providers, like AWS
and Azure, have traditionally relied on hiding the 
co-residency information to prevent targeted attacks 
in their clouds. But recent works have repeatedly 
found co-residence detection techniques that break
this encapsulation, prompting the providers to address 
them and harden isolation on their platforms. In 
this work, we find yet another such technique based on 
a memory bus covert channel that 
is more pervasive, reliable and harder to fix. We 
show that we can use this technique to reliably perform co-operative co-residence detection for thousands 
of AWS lambdas within a few seconds, which opens a 
way for attackers to perform DDoS attacks or learn 
cloud's internal mechanisms. We present this technique 
in detail, evaluate it and use it to perform a small
study on lambda activity across a few AWS regions. 
Through this work, we hope to motivate the need to 
address this covert channel in the cloud.

